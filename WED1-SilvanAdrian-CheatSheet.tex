% \documentclass[9pt,a4paper,twocolumn,landscape,oneside]{amsart}
\documentclass[8pt,a4paper,landscape,oneside]{amsart}
\usepackage{amsmath, amsthm, amssymb, amsfonts}
\usepackage[T1]{fontenc}
\usepackage[utf8]{inputenc}
\usepackage{booktabs}
\usepackage{fancyhdr}
\usepackage{float}
\usepackage{fullpage}
%\usepackage{geometry}
\usepackage[landscape]{geometry}
% \usepackage{listings}
\usepackage{caption, subcaption}
\usepackage[scaled]{beramono}
\usepackage{color,graphicx,overpic}
\usepackage{titling}
\usepackage{datetime}
\usepackage{multicol}
\usepackage{calc}
\usepackage{ifthen}
\usepackage{hyperref}
\usepackage{environ}
\usepackage{tabularx}
\usepackage[dvipsnames]{xcolor}


% Minted (For code stuff)
\usepackage{minted}
\newcommand{\code}[1]{\inputminted[fontsize=\footnotesize]{c}{#1}}

% This sets the margins to .5cm
\geometry{top=0pt,left=.3cm,right=.3cm,bottom=1cm}

\setlength{\headheight}{15.2pt}
\renewcommand{\headrulewidth}{0.4pt}
\renewcommand{\footrulewidth}{0.4pt}

% Turn off header and footer
% \pagestyle{empty}

% Redefine section commands to use less space
\makeatletter
\renewcommand{\section}{\@startsection{section}{1}{0mm}%
                                {-1ex plus -.5ex minus -.2ex}%
                                {0.5ex plus .2ex}%x
                                {\normalfont\large\bfseries}}
\renewcommand{\subsection}{\@startsection{subsection}{2}{0mm}%
                                {-1explus -.5ex minus -.2ex}%
                                {0.5ex plus .2ex}%
                                {\normalfont\normalsize\bfseries}}
\renewcommand{\subsubsection}{\@startsection{subsubsection}{3}{0mm}%
                                {-1ex plus -.5ex minus -.2ex}%
                                {1ex plus .2ex}%
                                {\normalfont\small\bfseries}}
\makeatother


% Don't print section numbers
\setcounter{secnumdepth}{0}


\setlength{\parindent}{0pt}
\setlength{\parskip}{0pt plus 0.5ex}

\newcommand{\subtitle}[1]{%
  \posttitle{%
    \par\end{center}
    \begin{center}\large#1\end{center}
    \vskip0.5em}%
}

% Header/Footer
% \geometry{includeheadfoot}
% \fancyhf{}
\pagestyle{fancy}
\lhead{HSR}
\rhead{Silvan Adrian\quad\thepage}
\cfoot{}

% Title/Author
\title{WED1 Cheat Sheet}
\subtitle{}
\date{\ddmmyyyydate{\today{}}}

% Output Verbosity
\newif\ifverbose
\verbosetrue
% \verbosefalse

% Some list helpers from graph theory
\newcounter{temp}
\newcounter{ilist_counter}
\newcounter{iilist_counter}

\newenvironment{ilist}{
  \begin{list}{{\bf \arabic{ilist_counter}}}{
      \usecounter{ilist_counter}
      \addtolength{\labelsep}{.6ex}
      \addtolength{\itemsep}{1ex}
      \setlength{\leftmargin}{1.4em}}
    %   \setcounter{ilist_counter}{\value{temp}}
}{
%   \setcounter{temp}{\value{ilist_counter}}
  \end{list}
}

\newenvironment{iilist}{
  \begin{list}{{\bf \alph{iilist_counter}}}{
      \usecounter{iilist_counter}
      \addtolength{\labelsep}{.6ex}
      \addtolength{\itemsep}{.5ex}
      \setlength{\leftmargin}{1.7em}}
}{
  \end{list}
}

\newenvironment{iblist}{
  \begin{list}{{\bf $\bullet$}}{
      \addtolength{\labelsep}{.6ex}
      \addtolength{\itemsep}{.5ex}
      \setlength{\leftmargin}{1.7em}}
}{
  \end{list}
}

% Theorems and solutions
\theoremstyle{plain}
\newtheorem{theorem}{Theorem}
\newtheorem*{theorem*}{Theorem}
\newtheorem{corollary}[theorem]{Corollary}
\newtheorem*{corollary*}{Corollary}
\newtheorem{lemma}[theorem]{Lemma}
\newtheorem*{lemma*}{Lemma}
\newtheorem{proposition}[theorem]{Proposition}
\newtheorem*{proposition*}{Proposition}
\newtheorem{conjecture}[theorem]{Conjecture}
\newtheorem*{conjecture*}{Conjecture}
\newtheorem*{solution}{Solution}

\theoremstyle{definition}
\newtheorem{definition}[theorem]{Definition}
\newtheorem*{definition*}{Definition}
\newtheorem{example}[theorem]{Example}
\newtheorem*{example*}{Example}
\newtheorem{problem}[theorem]{Problem}
\newtheorem*{problem*}{Problem}

\theoremstyle{remark}
\newtheorem{remark}{Remark}
\newtheorem*{remark*}{Remark}

% For writing vectors
\let\oldhat\hat
\let\oldvec\vec
\renewcommand{\vec}[1]{\oldvec{\mathbf{#1}}}
\newcommand{\vecb}[1]{\mathbf{#1}}
\renewcommand{\hat}[1]{\oldhat{\mathbf{#1}}}

\newcommand{\cvect}[2]{ \begin{pmatrix} #1 \\ #2 \end{pmatrix} }
\newcommand{\ctvect}[3]{ \begin{pmatrix} #1 \\ #2 \\ #3 \end{pmatrix} }
\newcommand{\vect}[2]{ \langle #1 , #2 \rangle }
\newcommand{\tvect}[3]{ \langle #1 , #2 , #3 \rangle }
\newcommand{\qvect}[4]{ \langle #1 , #2 , #3 \rangle }

% For equations
\NewEnviron{formula}{
    \abovedisplayshortskip=0pt
    \belowdisplayshortskip=0pt
    \abovedisplayskip=0pt
    \belowdisplayskip=0pt
    \begin{align*}
        \BODY
    \end{align*}
}
\newcommand{\eqn}[1]{\begin{formula} #1 \end{formula}}
% The actual document

\newcounter{line}
\newcolumntype{C}{>{\ttfamily\arraybackslash}l}
\newcolumntype{E}{>{\ttfamily\arraybackslash}c}
\newcolumntype{R}{>{\ttfamily\arraybackslash}r}
\newcolumntype{b}{>{\bfseries\arraybackslash}l}
\newenvironment{tabularlc}[1]{
\setcounter{line}{0}
\begin{tabular}{#1}
}{
\end{tabular}
}
\newenvironment{ldesc}{
\begin{tabularlc}{lC}
}{
\end{tabularlc}
}
\newenvironment{Ldesc}{
\begin{ldesc}
\hline
}{
\\\hline
\end{ldesc}
}
\newcommand{\C}{\texttt}
\newcommand{\B}{\textbf}
\newcommand{\I}{\textit}
\newcommand{\CI}[1]{\texttt{\textit{#1}}}
\newcommand{\N}{\textnormal}
\newcommand{\T}[1]{\hphantom{\I{#1}}}
\newcommand{\D}[1]{\hphantom{#1}}
\newcommand{\lditem}[2]{
	#1 & #2 \\
}
\newcommand{\li}[1][]{%
\stepcounter{line}%
\ifnum\theline>1 \\\fi%
#1 &
}
\newcommand{\lI}[1][]{%
\stepcounter{line}%
\ifnum\theline>1 \\[1ex]\fi%
#1 &
}
\newcommand{\Li}[1][]{%
\stepcounter{line}%
\ifnum\theline>1 \\\hline\fi%
#1 &
}
\newcommand{\LI}[1][]{%
\stepcounter{line}%
\ifnum\theline>1 \\[1ex]\fi%
#1 &
}
\newcommand{\s}{\hphantom{A}}
\newcommand{\comm}[1]{\textcolor{gray}{#1}}
\newcommand{\commi}[1]{\textit{\comm{#1}}}
\newcommand{\la}{\textbackslash}
\newcommand{\mtype}[1]{\multicolumn{2}{@{}C}{#1}}
\newcommand{\longtype}[1]{\multicolumn{3}{@{}C}{#1}}
\newcommand{\Longtype}[1]{\multicolumn{4}{@{}C}{#1}}
\newcommand{\longcall}[1]{\multicolumn{3}{@{}R@{\s$\equiv$\s}}{#1}}
\newcommand{\F}{\I{f}}
\newcommand{\X}{\I{x}}
\newcommand{\Y}{\I{y}}
\newcommand{\Z}{\I{z}}
\newcommand{\W}{\I{w}}
\newcommand{\XS}{\I{xs}}

\begin{document}
%\maketitle
\thispagestyle{fancy}
\raggedright
\footnotesize
\raggedcolumns
\begin{multicols*}{4}
\setlength{\premulticols}{1pt}
\setlength{\postmulticols}{1pt}
\setlength{\multicolsep}{1pt}
\setlength{\columnsep}{2pt}
\tiny

\subsubsection{HTML}

\textbf{HTML Document}
 \begin{minted}[frame=none,framesep=1mm,fontsize=\tiny]{html}
<!DOCTYPE html>
<html lang="en"> 
<head>
<title>First document</title>
<meta charset="utf­8" /> 
 <meta name="Author" content="Wikipraktika" />
  <meta name="Publisher" content="Wikipraktika" />
  <meta name="Description" content="Taschenrechner - 
  tragbare, handliche elektronische Rechenmaschine"/>
  <link rel="shortcut icon" type="image/gif" href="facepalm.gif" />
</head>
<body>
<h1>A first document</h1>
<p>This is our first HTML­document.</p> 
</body>
</html>
\end{minted}

\textbf{Formular}
 \begin{minted}[frame=none,framesep=1mm,fontsize=\tiny]{html}
<form action="url" method="post">
<fieldset><legend>Autor</legend>
<label>Herr <input type="radio" name="gender" value="Herr"></label>
<label>Name<input type="text" required=""></label>
<label>E-Mail<input type="email" required=""></label>
<label>Webseite</p><input type="url" required=""></label>
<label>Geburtsdatum</p><input type="date" required=""></label>
<label>Nationalität
<select required="">
<option value="Hello">Hello</option>
</select>
<label>
<label>Profiltext<textarea rows="10"></textarea></label>
<label>Profilbilder<input type="file"></label>
</fieldset>
<fieldset><legend>Rechtliches</legend>
<label><input type="checkbox" value="Hello World">
<span>Hello World</span></label>
</fieldset>
<button type="submit">Submit</button>
<button type="reset">Reset</button>
</form>
<datalist id="search-suggestions">
	<option value="Wings">Wings</option>
</datalist>
<input type="search" list="search-suggestions" />
\end{minted}

\subsubsection{CSS}
 \begin{minted}[frame=none,framesep=1mm,fontsize=\tiny]{html}
<link rel="stylesheet" href="all.css" type="text/css" media="all | 
    screen | print">
 \end{minted}
 
\begin{minted}[frame=none,framesep=1mm,fontsize=\tiny]{css}
/* Pseudeo Selecktoren */
:after, :before, :first-child, :last-child, :last-of-type, :only-of-type,
::first-line, ::first-letter {}
 /* Examples */
ul > li:nth-of-type(3n):not(:nth-of-type(even)){} /* nur alle direkten childs 
von ul */
a:not([href]){}
h1 + p{} /* Auf gleicher Basis */
h1 ~ p {} /* Styles werden auf alle p in dieser Ebene angewandt 
(wenn auch andere tags dazwischen)*/
body * h1{} /* Liegt nicht direkt im Body */
p:before {content:attr(data-foo) " ";} /* Use :attr to read out Attribute and 
set it in :before */
[data-tooltip]:hover:after{ content:  attr(data-tooltip);} /* Selector on 
Attribute */
[title~=besteht] {color: green;} /* Beinhaltet Alleinstehend'*/
* /* Beinhaltet generell */  $  /* Steht am ende */ 
| /* Startet Alleinstehend */  ^ /* Startet mit Wert */

\end{minted}

%usability
\subsubsection{Usability}
 \textbf{ Effektivität:}
Benutzer können ihre Ziele erreichen
\textbf{ Effizienz: }
Benutzer können ihre Ziele mit angemessenem Aufwand erreichen
\textbf{Zufriedenheit:}
Benutzer sind positiv gegenüber dem System eingestellt

\subsubsection{Prinzipien der Dialoggestaltung (ISO 9241-110)}

\textbf{Aufgabenangemessenheit}
Benutzer erledigen Aufgaben effektiv und effizient
\textbf{Selbstbeschreibungsfähigkeit}
Dialogschritte sind durch Beschreibungen oder Rückmeldungen verständlich oder werden erklärt
\textbf{Steuerbarkeit}
Benutzer kann Richtung und Geschwindigkeit der Interaktion beeinflussen
\textbf{Erwartungskonformität}
Dialog entspricht den Kenntnissen des Benutzers
\textbf{Fehlertoleranz}
Trotz erkennbar fehlerhafter Eingaben des Benutzers kann das Ziel effizient erreicht werden
\textbf{Individualisierbarkeit}
Interaktion kann angepasst werden
\textbf{Lernförderlichkeit}
Erlernen der Anwendung wird unterstützt
\subsubsection{Stone et al. 2005}
\textbf{Visibility:} Der erste Schritt zum Ziel ist sichtbar
\textbf{Affordance:} (Begreifbarkeit) Aktionsresultat ist vorhersagbar
\textbf{Feedback:} Es ist klar was passiert ist (oder passiert -> Animation)
\textbf{Simplicity:} Nicht mehr als nötig für die Aufgabe
\textbf{Structure:} Logische und konsistente Organisation
\textbf{Consistency:} Vorhersagbarkeit durch Konsistenz
\textbf{Tolerance:} Fehler vermeiden, Wiederherstellung vereinfachen
\textbf{Accessibility:} Design für alle Personengruppen \& Situationen
\subsubsection{Garret}
\textbf{Oberfläche:} konkret Attraktiv, Vertrauenserweckend, .. “Hedonische Qualität“
-> Graphic Design
\textbf{Raster:} Erwartungskonform, Fehlertolerant, Effizient -> Interaktionsdesign mit Wireframes
\textbf{Struktur} Aufgabengerecht, Fehlertolerant, Effizient
-> Interaktionsdesign mit Szenarios -> Informationsarchitektur
\textbf{Umfang} Zielgruppengerecht, Effektivität -> Anforderungsanalyse, Ziele, Aufgaben
\textbf{Strategie}  Marktgerecht-> Marktanalyse, Zielgruppenanalyse
%Usability
%UCD
\subsubsection{UCD}
\subsubsection{Cognitive Walktrough / Usability Testing}
\tiny Ist eine je Methode zur Identifikation von Usability Problemen. Cognitive 
Walktrough wird durchgeführt von 3 oder mehr Experten und Usability Testing 
von 3 oder mehr zukünftigen Nutzer. Für Usability Testing braucht es noch kein 
lauffähiges System, Screenshots reichen. Grösste Herausforderung bei Usability 
Testing ist die Erstellung guter Aufgaben, bei welchen echte Benutzerziele getestet 
werden ohne, dass Keywords verraten werden.
\subsubsection{User Centered Design}
Elemente der User Experience: \textbf{Oberfläche} (Aufmerksamkeiten, 
Erkennung von Bekanntem), \textbf{Raster} (Responsive Design), \textbf{Struktur} (Usability), 
\textbf{Umfang} , \textbf{Strategie}.

%UCD
\subsubsection{JS}
 \begin{minted}[frame=none,framesep=1mm,fontsize=\tiny]{javascript}
 let arr = [3, 5, 7];

for (let i in arr) {
   console.log(i); // logs "0", "1", "2", "foo" z.B.: for JSON
}

for (let i of arr) {
   console.log(i); // logs "3", "5", "7"
}
/* Funktionsübergabe an Function */
function clock(seconds, action){
    setTimeout(action, seconds * 1000);
    console.log(`timer goes off in: ${seconds} seconds`);
}
clock(30, () => console.log("Hallo World"));
/* Counter Function */
function counter(start){
    return (
    {
        incr : () =>  ++start,
        decr : () =>  --start,
        value : () => start
    });
}
var c1 = counter(2);
console.log("3:", c1.incr());
console.log("4:", c1.incr());
console.log("5:", c1.incr());
console.log("5:", c1.value());
console.log("4:", c1.decr());
//NaN (has Type Number)
NaN == Nan //false check with isNan(var)
parseFloat("1.2ab") //1.2
parseInt("abc") //NaN
var filteredArray = array.filter( elem => elem > 5); 
console.log(array.every( elem => elem > 5));
 \end{minted}

\subsubsection{DOM - Events - jQuery}
 \begin{minted}[frame=none,framesep=1mm,fontsize=\tiny]{javascript}
 document.querySelectorAll("h2")
 window.addEventListener("load", function(){
document.querySelector("input[value='Hello World']").
addEventListener("click", function(){ console.log("File", "Hello World");});
});
var currencies = Array.prototype.slice.call
(document.querySelectorAll("[data-currency]"));
currencies.forEach(function(el) {
    el.addEventListener("input",function(input){
        var currentElement = this; // or input.target;
        currencies.forEach(function(toChange){
            if(toChange!=currentElement){
                toChange.value = x ;
            }
        });
    });
}); 
/* keypress Event (down, up)*/
document.getElementById('inpt').onkeypress = function (event) {
    var self = this;
    setTimeout(function() {
        self.value = self.value.toUpperCase()
    }, 0)
};
/* Set innerHTML */
document.getElementById("songs").innerHTML=createSongsHtml(songs);
\end{minted}

\subsubsection{jQuery Selektoren}

 \begin{minted}[frame=none,framesep=1mm,fontsize=\tiny]{javascript}
 (function($) { })(jQuery);  //Vorgabe für NoConflict
$('#head > title').text() );
$('#td th').first().text() ); //first in #td th
print( $('tr').length ); //count tr's
print( $('#age').val() + 10 ); // add text 10 to val
function() { if($('#sex').val()) { return "t"; } else { return "f"; } }()
typeof($('#age2').val()-7) //number
typeof($('#sex').html) //function
$('#th > td').text() //Concat Text that belong to selector
//jQuery Function Chaining
$('#lblFormHeader').blur().hide();
 \end{minted}
 
 \subsubsection{jQuery Events}
  \begin{minted}[frame=none,framesep=1mm,fontsize=\tiny]{javascript}
$( '#target' ).click(function() {
    alert( "Handler for .click() called." );
});
$('.haschibaschi').on('click | submit | event' , function() {});
//reference Event handler
$(".number").on("click", buttonNumClickHandler);
function buttonNumClickHandler (event) {
   input.text(input.text() + event.target.value);
}
\end{minted}

\subsubsection{HTTP - Node}
\textbf{Etag}: Im HTTP Header nach dem ersten Request wird dieser eingefügt, zur 
Abklärung, ob das Bild bereits gecacht wurde 
(bei einem weiteren Request und falls sich die Datei nicht geändert hat seit dem letzten 
Aufruf wird ein 304 not Modified zurückgeschickt)

\subsubsection{AJAX}
\begin{minted}[frame=none,framesep=1mm,fontsize=\tiny]{javascript}
//XHTMLHttpRequest GET (Example with getting time)
// create new AJAX request
var request = new XMLHttpRequest();
// handle successfull response
request.onreadystatechange = function() {
    if (request.readyState == 4 && request.status == 200) {
    } };
// do something with 'request.responseText'
window.alert(request.responseText); //JSON.parse(client.responseText)['time']
// fire asynchronous AJAX request (true is default)
request.open('GET', 'http://domain.tld/product-details', true); 
request.send();
client.setRequestHeader('Accept','application/json'); //JSON
request.send();
//Post
var data = "id=96&category=6";
request.open("POST", url);
// Content type necessary
request.setRequestHeader( "Content-Type",
    "application/x-www-form-urlencoded"
); 
request.send(data);
request.send(JSON.stringify(data)); //Send JSON
 \end{minted}
 
 \subsubsection{jQuery AJAX}
 
\begin{minted}[frame=none,framesep=1mm,fontsize=\tiny]{javascript}
  //Form Submit
$(document).ready(function () {
    var form = $('#contactForm');
    // listen on form submit
    form.on('submit', function(event) {
        // send form data serializes (url encoded) by AJAX
        $.post(form.attr('action'), form.serialize(), function(data) {
            $('#status').get(0).innerHTML = data;
            form.get(0).reset();
        });
        // disable form default action
        event.preventDefault(); });
});
//$.ajax
$.ajax({
  type: "POST/GET/LOAD",
  url: url,
  data: data,
  success: success,
  dataType: dataType
});
//jQuery Get Request
$.get( "ajax/test.html", function( data ) {
    $( ".result" ).html( data );
    alert( "Load was performed." );
});
//JQuery POST
$.post( "ajax/test.html", function( data ) {
  $( ".result" ).html( data ); //JSON.stringify(data)
});
//jQuery GetJSON
$.getJSON( "/time").done(function(data)
{onTimeAvailable(data['time']);}).fail(onTimeError);
//jQuery load
$( "#result" ).load( "ajax/test.html", function() {
    alert( "Load was performed." );
});
\end{minted}


\subsubsection{Template Engine}
\begin{minted}[frame=none,framesep=1mm,fontsize=\tiny]{html}
  {{#if books}}
	<p>{{books.length}} Bücher vorhanden</p>
{{else}}
	<p>keine Bücher gefunden</p>
{{/if}
{{#each books}}
  <p>{{title}}</p>
{{/each}}
\end{minted}

\begin{minted}[frame=none,framesep=1mm,fontsize=\tiny]{javascript}
 //Handlebars
var appContainer = $('#app-container');

function applyTemplate(templateSource, data) {
	var template = Handlebars.compile(templateSource);
	var view = template(data);
	appContainer.html(view);
}

function renderResult(data) {
	$.get(config.resultTemplateUrl, function(resultTemplateSource) {
		applyTemplate(resultTemplateSource, data);
	});
}
//Handlebars Example
var viewRenderer = Handlebars.compile(productTemplate);
var view = viewRenderer({ products: products });
//Simple Template Engine without Handlebars
function applyTemplate(template, dataObject) {
    template.find('p').first().html("<p>Id: " + dataObject.id + "</p>");
    template.find('p').last().html("<p>Name: " + dataObject.name + "</p>");
}
$("#add").click(function(){
    var data = {
        id: $("#id").val(),
        name: $("#name").val()
    };
    var template = $("#template");
    applyTemplate(template,data);
});
//Register Handelbar Helper
Handlebars.registerHelper('date', function(date, format) {
	return moment(date).format(format);
});
\end{minted}




\end{multicols*}
\end{document}